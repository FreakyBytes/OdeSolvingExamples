\documentclass{article}

\usepackage{german}
%\usepackage[ngerman]{babel}
\usepackage[utf8]{inputenc}

\usepackage{mathtools}
\usepackage{amssymb}
\usepackage{listings}
\usepackage{array,ragged2e}
\usepackage{color}
\usepackage{hyperref}

\usepackage{graphicx}

%Umlaute für Listings definieren
\lstset{
  literate= {Ö}{{\"O}}1 {Ä}{{\"A}}1 {Ü}{{\"U}}1 {ß}{{\ss}}2 {ü}{{\"u}}1
 {ä}{{\"a}}1 {ö}{{\"o}}1
 }

% Dokument

\title{Mathe Hausaufgaben - Serie 13}

\author{Martin Peters (212206972)}
\date{09. Juli 2013}

\begin{document}
	\maketitle
	
	Die gegebenen Werte wurde mit Hilfe eines selbst geschriebenen Java-Programms berechnet. Der Sourcecode ist unter \href{https://github.com/FreakyBytes/OdeSolvingExamples}{https://github.com/FreakyBytes/OdeSolvingExamples} frei einsehbar. Fehler durch unzureichende Genauigkeit des benutzten Variablentypes (8 byte floating point value) sind möglich.	\\ \\
	
	\begin{tabular}{|c|c|c|c|c|c|}
		\hline
		\(i\) & \(x_i\) & \(y_i\) Eulerpolygonzug & \(\tilde{y_i}\) & \(y_i\) Euler-Heun & \(y_i\) Range-Kutta \\
		\hline 
		0  & \(1.0\) & \(0.000000\) & \(0.300000\) & \(0.000000\) & \(0.000000\) \\ \hline
		1  & \(1.1\) & \(0.300000\) & \(0.523140\) & \(0.272727\) & \(0.273552\) \\ \hline
		2  & \(1.2\) & \(0.545455\) & \(0.720282\) & \(0.504339\) & \(0.505553\) \\ \hline
		3  & \(1.3\) & \(0.754545\) & \(0.898150\) & \(0.706904\) & \(0.708281\) \\ \hline
		4  & \(1.4\) & \(0.938462\) & \(1.061463\) & \(0.888373\) & \(0.889793\) \\ \hline
		5  & \(1.5\) & \(1.104396\) & \(1.213600\) & \(1.054154\) & \(1.055553\) \\ \hline
		6  & \(1.6\) & \(1.257143\) & \(1.357024\) & \(1.208027\) & \(1.209373\) \\ \hline
		7  & \(1.7\) & \(1.400000\) & \(1.493559\) & \(1.352700\) & \(1.353977\) \\ \hline
		8  & \(1.8\) & \(1.535294\) & \(1.624582\) & \(1.490154\) & \(1.491356\) \\ \hline
		9  & \(1.9\) & \(1.664706\) & \(1.751141\) & \(1.621864\) & \(1.622990\) \\ \hline
		10 & \(2.0\) & \(1.789474\) & \(1.874051\) & \(1.748945\) & \(1.749998\) \\ \hline
	\end{tabular}

\end{document}
